%----------
% \section{ }
% \subsection{ }\label{ }\index{ }
%----------

%----
% \section{ }
% \subsection{ }\label{ }\index{ }
%----

\section{Einschränkungen der Standardlizenz von Adobe
Stock}\label{einschraenkungen-der-standardlizenz-von-adobe-stock}

Unter einer Standardlizenz zulässig:

\begin{itemize}% Liste Punkt% Liste Punkt
\item
  Reproduktion von bis zu 500.000 Kopien des Stockmediums auf
  Produktverpackungen und in gedruckten Marketingmaterialien sowie in
  digitalen Dokumenten oder Software.
\item
  Einschließen des Stockmediums in E-Mail-Marketing, mobile Werbung oder
  Fernsehprogramme, wenn weniger als 500.000 Zuschauer zu erwarten sind.
\item
  Veröffentlichen des Stockmediums auf einer Website ohne Einschränkung
  der Anzahl an Besucher. Wenn das Stockmedium unverändert auf einer
  Website der sozialen Medien veröffentlicht wird, ist eine
  Namensnennung erforderlich ((c) Autorenname -- stock.adobe.com).
\item
  Einschließen des Stockmediums in Produkte auf geringfügige Weise, z.
  B. in Lehrbüchern.
\end{itemize}

Unter einer Standardlizenz nicht zulässig:

Erstellen von Waren oder Handels- und Vertriebsprodukten, bei denen das
Stockmedium selbst den Hauptwert des Produkts ausmacht. So dürfen Sie
beispielsweise keine Poster, T-Shirts oder Kaffeetassen gestalten, die
hauptsächlich wegen des aufgedruckten Stockmediums gekauft werden.

Bilder

((C) Autorenname -- stock.adobe.com).

((C) Sergey Nivens -- stock.adobe.com).
%-----
%-----------
