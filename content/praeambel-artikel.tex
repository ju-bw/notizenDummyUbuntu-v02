% Latex: ju - 20-Nov-18 - praeambel.tex 
\usepackage[T1]{fontenc}               % Schriftarten
\usepackage[english,ngerman]{babel}    % dt. Silbentrennung
\usepackage[utf8]{inputenc}            % Kodierung
% Schrift
\usepackage[osf,sc]{mathpazo}
\usepackage[scale=.9, semibold]{sourcecodepro}
\usepackage[osf]{sourcesanspro}
%\usepackage{lmodern}
%
\usepackage[table,x11names,dvipsnames,usenames]{xcolor}% Farben
\usepackage{setspace}
\usepackage{graphicx}
\usepackage{subcaption}
\usepackage{amsmath}
\usepackage{amssymb}%Pfeile
\usepackage{mathtools}% lädt amsmath und korrigiert zwei Fehler
\usepackage{siunitx} % \num{12345,678999} 12 345.678 999
\usepackage{color}
\usepackage{multicol}
\usepackage{framed}
\usepackage{wrapfig}
\usepackage{float}
\usepackage{fancyhdr}
\usepackage{verbatim}
\usepackage{tcolorbox}
\usepackage{lipsum}
\usepackage{blindtext}
\usepackage{tocloft}


% bibliography
\usepackage[babel,german=guillemets]{csquotes} %deutsches Anführungszeichen
\usepackage[style=ieee, bibencoding=utf8, backend=biber]{biblatex} % biblatex mit biber laden
\ExecuteBibliographyOptions{
	backref=false,
	backrefstyle=three+,
	url=true,
	urldate=comp,
	abbreviate=false,
	maxnames=20
}

\usepackage{enumitem}% Definition neuer Listentypen
\usepackage{tabularx}% Tabellen mit flexibler Spaltenbreite
\usepackage{booktabs}% schönere Tabellenlinien
\usepackage{longtable}
\usepackage{rotating}
\usepackage{listingsutf8}
\usepackage{url}% Trennung an vernünftigen Stellen
\usepackage{pdfpages}% PDF-Dokumente einbinden
\usepackage{qrcode}% Anwendung: \qrcode[hyperlink,height=5cm]{/}

% Abb., Tab., Prog.
\renewcaptionname{ngerman}{\figurename}{Abb.}
\renewcaptionname{ngerman}{\tablename}{Tab.}
\renewcommand{\lstlistingname}{Prog.}


%% Farben vordefiniert:
% white lightgray gray darkgray black
% yellow red green blue
% cyan magenta teal
% brown lime olive orange
% pink purple violet

%% eig. Farbe
\definecolor{hellesbrombeer}{rgb}{0.8,0.4,0.8}
\definecolor{meinred}{rgb}{0.9, 0.13, 0.13}
\definecolor{meinblue}{rgb}{0,0.38671875,0.64453125}
\definecolor{meingreen}{rgb}{0.0, 0.34, 0.25}
\definecolor{meinorange}{rgb}{1.0, 0.55, 0.0}
\definecolor{meingrey}{rgb}{0.953125,0.96484375,0.98046875}
\definecolor{meinpink}{rgb}{255, 0, 102}
\definecolor{darkblue}{rgb}{0.03, 0.27, 0.49}
\definecolor{darkred}{rgb}{153, 9, 9}
% \colorbox{red!10!white} % Tabelle 10% rot, Rest weiß}

%% Kapitelüberschriften farbig
\usepackage{sectsty}
\chapterfont{\color{red!75!black}}
\sectionfont{\color{red!75!black}}
\subsectionfont{\color{red!75!black}}
\subsubsectionfont{\color{red!75!black}}


%% Quellcode
\usepackage{listingsutf8}
\lstset{%
	basicstyle=\small\ttfamily, %Schriftformat  \texttt{Maschinenschrift},
	showstringspaces=false,
	%numbers=left,
	numberstyle=\tiny,
	numbersep=5pt,
  %stepnumber=2,      %Jede zweite Zeile nummerieren
	%backgroundcolor=\color{meingrey},	%helles grau
	showspaces=false,        % show spaces adding particular underscores
	showstringspaces=false,  % underline spaces within strings
	showtabs=false,          % show tabs within strings adding particular underscores
	%frame=false,            % adds a frame around the code
	tabsize=2,               % Tabulator
	breaklines=true,         % Zeilen umbrechen wenn notwendig.
	breakautoindent=true,    % Nach dem Zeilenumbruch Zeile einrücken.
	numberblanklines=false,
	postbreak=\space,        % Bei Leerzeichen umbrechen.
	resetmargins=true,
	gobble=2,
  captionpos=b,            % sets the caption-position to bottom or top
	title=,                  % show the filename of files included with \lstinputlisting;
	prebreak=\mbox{ $\curvearrowright$},%code umbruch
	linewidth=\columnwidth,
	keywordstyle=\color{red!75!black},% Schlüsselwörter
	stringstyle=\color{meinorange},   % Variablen
	commentstyle=\color{meingreen},   % Kommentare
	emphstyle=\color{darkblue},       % Variablen
	%morekeywords={subsection},
	%language=[LaTeX]TeX % Sprache
}
\lstset{literate=%
	{Ö}{{\"O}}1
	{Ä}{{\"A}}1
	{Ü}{{\"U}}1
	{ß}{\ss}2
	{ü}{{\"u}}1
	{ä}{{\"a}}1
	{ö}{{\"o}}1
	{»}{{\frqq}}4
	{«}{{\flqq}}4
}




\usepackage[breaklinks=true]{hyperref}	% 2. Hyperlinks und Lesezeichen in PDF
\usepackage[ngerman]{cleveref}          % 3. Automatische Querverweise

%% PDF-Meta-Informationen und Darstellung von Links im Dokument
\hypersetup{
	draft =false,
	colorlinks,
	linkcolor={red!75!black}, % Inhaltsverzeichnis farbe
	citecolor={red!75!black},
	filecolor={red!75!black},
  pagecolor={red!75!black},
	urlcolor={darkblue},
	bookmarksopen=true, bookmarksopenlevel=1,
	bookmarks=true,           % show bookmarks bar?
  unicode=true,             % non-Latin characters in Acrobats bookmarks
  pdftoolbar=true,          % show Acrobats toolbar?
  pdfmenubar=true,          % show Acrobats
  pdffitwindow=false,       % window fit to page when opened
  pdfstartview={FitH},      % fits the width of the page to the window
  pdftitle=Thema,           % title
  pdfauthor=Jan Unger,      % author
  pdfsubject=Mitschrift,    % subject of the document
  pdfcreator=LaTeX,         % creator of the document
  pdfproducer=Koma,         % producer of the document
  pdfkeywords=Schlagwoerter,% list of keywords
  pdfnewwindow=true,        % links in new windowrgb(210, 35, 42)
	hyperfootnotes=true,      % Links auf Fußnoten
	hyperindex=true,          % Indexeinträge verweisen auf Text
	linkbordercolor={0 1 1},  % Rahmenfarbe interne Links
	menubordercolor={0 1 1},  % Rahmenfarbe Literaturlinks
	urlbordercolor={1 0 0}    % Rahmenfarbe externe Links
}

%% Mathe linksbündig: \documentclass-option: fleqn
\setlength{\mathindent}{5mm} % Mathe-Einrücktiefe

\onehalfspacing             % Zeilenabstand 1,5

\setlength{\parindent}{0cm} % Einrücken der ersten Zeile, Absatz

\usepackage[margin=2.5cm]{geometry}

%% Kapitelnummer und Kapitelname - Abstand
%\renewcommand*\chapterformat{\thechapter.~\vspace{5mm}}

\pagestyle{fancy}
\lfoot{\textbf{\titel}}
\rfoot{Seite \thepage}
\lhead{\textbf{\leftmark}}
\rhead{\textbf{\rightmark}}
\cfoot{}
\renewcommand{\footrulewidth}{0.5pt}
\renewcommand{\headrulewidth}{0.5pt}
\doublespacing


\makeatletter

% ++++++++++++++++++++++++++++++++++
%% Textauszeichnung
% \emph{kursiv}
% \textrm{Antiqua}, \textsf{Grotesk}, \texttt{Maschinenschrift},
% \textmd{normal}, \textbf{breiter}, \textup{aufrecht}, \textsl{geneigt},
% \textit{kursiv}, \textsc{Kapitaelchen}

%% Schriftgroesse
% \tiny{winzig}, \scriptsize{sehr klein}, \footnotesize{klein},
% \small{klein}, \normalsize{normal}, \large{gross}, \Large{groesser},
% \LARGE{ganz gross}, \huge{riesig}, \Huge{gigantisch}

%% eigene Befehle definieren
% Textauszeichnung: \wort{Beispiel}, \fremdwort{prezioes}
\newcommand{\wort}[1]{\emph{#1}}
\newcommand{\fremdwort}[1]{\textsf{#1}}

%% Textabstand:  Verwendung: \abstand{}
\newcommand{\abstand}[1]{\vspace{5mm}{#1}}
%% quad, qquad, hspace{20mm}, vspace{20mm}
%
% Wichtig (Optionale Parameter)
%% Wort Kursiv u. in Farbe
\newcommand{\wichtig}[2][red]{\textcolor{#1}{\emph{#2}}}

% Eigene Umgebung
% Verwendung: \begin{hinweis}Ein Text.\end{hinweis}
\newenvironment{hinweis}[1][Hinweis]{%
  \begin{quote}
  \color{meinblue}\rule{0.87\textwidth}{1pt}\\%
	\color{black}
  \textbf{#1:}\\ %
}{%
	\vspace{1mm}
  \\\color{meinblue}\rule[5ex]{0.87\textwidth}{1pt}%
  \end{quote}
}

% farbige Infobox
% Anwendung:
% \myInfoBox{Text}
\newcommand\myInfoBox[1]{%
  \begin{quote}
	\fcolorbox{meinblue}{meingrey}{%
		\parbox{0.85\textwidth}{%
			\textbf{Hinweis:}\\%
			#1
		}
	}
\end{quote}
}

% farbige Infobox 2
% Anwendung:
% \mybox{Text}
\newcommand{\mybox}[1]{%
  \setbox0=\hbox{#1}%
  \setlength{\@tempdima}{\dimexpr\wd0+13pt}%
  \begin{tcolorbox}[colframe=meinblue,boxrule=0.5pt,arc=4pt,
      left=6pt,right=6pt,top=6pt,bottom=6pt,boxsep=0pt,width=0.95\textwidth]
    #1
  \end{tcolorbox}
}

% farbige Listenbox
% Anwendung:
%\myListenBox {
%	\item Listenpunkt
%	\item Listenpunkt
%	\item Listenpunkt
%}
\newcommand\myListenBox[1]{%
	\begin{quote}
		\fcolorbox{meinblue}{white}{%
			\parbox{0.85\textwidth}{%
				% Inhalt
				\textbf{Liste: }
				\begin{itemize}[label=$\square$]%checkbox
					#1
				\end{itemize}
			}
		}
	\end{quote}
}
