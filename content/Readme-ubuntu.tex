%----------
% \section{ }
% \subsection{ }\label{ }\index{ }
%----------

\section{Readme}\label{readme}

Erstellt Websiten \& Latex PDFs mit Markdown und pandoc.

Sed passt die Syntax annotizen

Versionsverwaltung: git

\subsection{Hinweis}\label{hinweis}

Projekt getestet unter Ubuntu 18.04.2 LTS.

\subsection{Projekt erstellen}\label{projekt-erstellen}

Das Script \frqq pdfname-umbenennen.sh\flqq\ sucht und ersetzt den pdfnamen.

ACHTUNG: Script außerhalb vom neu-notiz-proj ausführen.

% Quellcode
\lstset{language=Bash} % C, [LaTeX]TeX, Bash, Python
\begin{lstlisting}[numbers=left, frame=l, framerule=0.1pt,%
% -----------
	caption={ }, % Caption
	label={code: }  % Label
]% ---------=

  # Shell
  # Repository  von Github downloaden
  $ git clone git@github.com:ju-bw/notizenDummyUbuntu-v02.git neu-notiz-proj/
  $ cp neu-notiz-proj/pdfname-umbenennen.sh .

  # Script anpassen
  $ vi pdfname-umbenennen.sh
  $ ./pdfname-umbenennen.sh

  $ cd neu-notiz-proj/
  $ ./projekt.sh
\end{lstlisting}

\subsection{Software}\label{software}

Pandoc: \url{https://pandoc.org/installing.html}

Latex: \url{https://www.tug.org/texlive/acquire-netinstall.html}

% Quellcode
\lstset{language=Bash} % C, [LaTeX]TeX, Bash, Python
\begin{lstlisting}[numbers=left, frame=l, framerule=0.1pt,%
% -----------
	caption={ }, % Caption
	label={code: }  % Label
]% ---------=

  # Shell
  # TeXlive update
  $ tlmgr update --all
\end{lstlisting}

Editor:

\url{https://code.visualstudio.com/download}

\url{https://atom.io/}

Git: \url{https://git-scm.com/downloads}

% Quellcode
\lstset{language=Bash} % C, [LaTeX]TeX, Bash, Python
\begin{lstlisting}[numbers=left, frame=l, framerule=0.1pt,%
% -----------
	caption={ }, % Caption
	label={code: }  % Label
]% ---------=

  # Shell
  # Git version
  $ git --version
\end{lstlisting}

Imagemagick:
\url{https://www.imagemagick.org/script/download.php\#windows}

\subsection{Repository von Github
downloaden}\label{repository-von-github-downloaden}

Repository = notizenDummyUbuntu-v02.git

% Quellcode
\lstset{language=Bash} % C, [LaTeX]TeX, Bash, Python
\begin{lstlisting}[numbers=left, frame=l, framerule=0.1pt,%
% -----------
	caption={ }, % Caption
	label={code: }  % Label
]% ---------=

  # Shell
  # Github Repository downloaden
  $ git clone git@github.com:ju-bw/notizenDummyUbuntu-v02.git .
  # oder
  $ git clone https://github.com/ju-bw/notizenDummyUbuntu-v02.git .
  # lokales backup repository
  $ git clone /media/jan/virtuell/git-server-repo/notizenDummyUbuntu-v02-backup.git .
\end{lstlisting}

\subsection{Neues Repository auf Github
anlegen}\label{neues-repository-auf-github-anlegen}

\url{https://github.com/new}

Create a new repository

Repository name = notizenDummyUbuntu-v02

% Quellcode
\lstset{language=Bash} % C, [LaTeX]TeX, Bash, Python
\begin{lstlisting}[numbers=left, frame=l, framerule=0.1pt,%
% -----------
	caption={ }, % Caption
	label={code: }  % Label
]% ---------=

  # Voraussetzung:
  #
  # lokales Repository: HEAD -> master
  git init # rm -rf .git
  git commit -am "Projekt init"
  #
  # Github Repository: origin/master
  adresse="github.com:ju-bw"
  git remote add origin git@$adresse/notizenDummyUbuntu-v02.git
  git push -u origin master
  #
  # lokales backup Repository: backup/master
  SSD="/media/jan/virtuell/git-server-repo"
  git clone --no-hardlinks --bare . $SSD/notizenDummyUbuntu-v02-backup.git
  git remote add backup $SSD/notizenDummyUbuntu-v02-backup.git
  git push --all backup
\end{lstlisting}

% Quellcode
\lstset{language=Bash} % C, [LaTeX]TeX, Bash, Python
\begin{lstlisting}[numbers=left, frame=l, framerule=0.1pt,%
% -----------
	caption={ }, % Caption
	label={code: }  % Label
]% ---------=

  # Shell: Git Befehle
  #
  # ".gitconfig", ".gitignore" erstellen und konfigurieren
  #
  # git versionieren
  git add .
  git commit -a # Editorauswahl: sudo update-alternatives --config editor
  git status
  git log --graph --oneline

  # github repository
  git status
  git pull
  git push
  git log --graph --oneline

  # lokales backup repository
  git push --all backup # sichern
  git status
  git log --graph --oneline

  # branch erstellen
  git checkout -b work
  git checkout work
  # projekt bearbeiten
  git checkout master
  git merge work

  git status
  git log --graph --oneline # beenden q
  git log --graph --pretty=format:";  %cn;  %h;  %ad;  %s" --date=relative > $file
\end{lstlisting}

\subsection{Markdown Dokumente / Notizen
verfassen}\label{markdown-dokumente-notizen-verfassen}

Markdown Dokumente / Notizen im Ordner \frqq md/neu.md\flqq\ erstellen.

% Quellcode
\lstset{language=Bash} % C, [LaTeX]TeX, Bash, Python
\begin{lstlisting}[numbers=left, frame=l, framerule=0.1pt,%
% -----------
	caption={ }, % Caption
	label={code: }  % Label
]% ---------=

  # Markdown

  <!--ju - Letztes Update: 6-Apr-19  -->

  ## Quellcode

  (\autoref{code: } ). % Codeverweis = Codename
\end{lstlisting}

% Quellcode
\lstset{language=Bash} % C, [LaTeX]TeX, Bash, Python
\begin{lstlisting}[numbers=left, frame=l, framerule=0.1pt,%
	% -----------
		caption={ }, % Caption
		label={code: }  % Label
	]% ---------=

	# Überschrift
	## Überschrift 2
	### Überschrift 3
\end{lstlisting}

% Quellcode
\lstset{language=Bash} % C, [LaTeX]TeX, Bash, Python
\begin{lstlisting}[numbers=left, frame=l, framerule=0.1pt,%
% -----------
	caption={ }, % Caption
	label={code: }  % Label
]% ---------


	## Bild

	Bilder in pdf speichern, notwendig für Latex.

	% Bild Referenz
	(\autoref{pic: } ). % Bildverweis = logo.pdf

	![Logo](conent/logo.pdf)

	![Bild](https://cdn.pixabay.com/photo/2019/04/02/04/32/masala-4096891_960_720.jpg)

	## Tabelle

	(\autoref{tab: } ). % Tabellenverweis = table

	|**Nr.**|**Begriffe**|**Erklärung**|
	|------:|:-----------|:------------|
	| 1     | a1         | a2          |
	| 2     | b1         | b2          |
	| 3     | c1         | c2          |
\end{lstlisting}

\textbf{Scripte \frqq projekt.sh\flqq\ und \frqq scripte/sed.sh\flqq\ anpassen}

% Quellcode
\lstset{language=Bash} % C, [LaTeX]TeX, Bash, Python
\begin{lstlisting}[numbers=left, frame=l, framerule=0.1pt,%
% -----------
	caption={ }, % Caption
	label={code: }  % Label
]% ---------=

  # Shell
  $ cd neu-notiz-proj
  # Script anpassen
  $ vi scripte/sed.sh
    # file
    # codelanguage
    scripte/sed.sh  <- HTML5, Python, Bash, C, C++, [LaTeX]TeX

    # CMS server pfad
    scripte/sed.sh  <- https://www.ju1.eu/*

    scripte/sed.sh  <-  bildformat
        pdf           -> latex
        svg, png, jpg -> web
  $ vi projekt.sh
    # file
    # Titel -> ../pdfname-umbenennen.sh
    pdfname="notizenDummyUbuntu-v02"
    # Backup
    SSD="/home/jan/Downloads"
    backup="$SSD/backup/notizen"
\end{lstlisting}

\textbf{Script ausfuehren}

% Quellcode
\lstset{language=Bash} % C, [LaTeX]TeX, Bash, Python
\begin{lstlisting}[numbers=left, frame=l, framerule=0.1pt,%
% -----------
	caption={ }, % Caption
	label={code: }  % Label
]% ---------=

  # Shell
  $ cd neu-notiz-proj
  $ ./projekt.sh

  Projekt Web & Latex Ubuntu

  0) Projekt erstellen.
  1) Markdown in (tex, html5) - sed (Suchen/Ersetzen)
  2) Kopie tex (Pandoc) - tex (Handarbeit)
  3) Kapitel erstellen, Scripte ausführen
  4) TEST: PDF erstellen mit pdflatex (book.pdf)
  5) TEST: PDF erstellen mit latexmk (light.pdf)
  6) PDFs erstellen (book-, print-, artikel.pdf) - Archiv (tex)
  7) Projekt aufräumen
  8) Git-Version erstellen
  9) git status und git log --graph --oneline
  10) git init
  11) Fotos optimieren (Web, Latex)
  12) PDF-Versionen erstellen
  13) Backup (archiv/*.zip & *.tar.gz) & (/media/jan/virtuell/backup)
  14) Beenden?

  Geben Sie eine Zahl ein:
\end{lstlisting}

\subsection{Bilder optimieren}\label{bilder-optimieren}

\textbf{JPG Bilder} in den Ordner \frqq img-in/\flqq\ kopieren.

optimiert Fotos für das Web und die PDF Datei.

\subsection{Backup}\label{backup}

% Quellcode
\lstset{language=Bash} % C, [LaTeX]TeX, Bash, Python
\begin{lstlisting}[numbers=left, frame=l, framerule=0.1pt,%
% -----------
	caption={ }, % Caption
	label={code: }  % Label
]% ---------=

  # Shell
  $ cd neu-notiz-proj
  $ tar cvzf ../notizenDummyUbuntu-v02.tar.gz .
\end{lstlisting}

\section{Git Version
Wiederherstellen}\label{git-version-wiederherstellen}

\subsection{Wiederherstellen}\label{wiederherstellen}

\subsubsection{Ordner für Experimente erstellen -
löschen}\label{ordner-fuer-experimente-erstellen-loeschen}

% Quellcode
\lstset{language=Bash} % C, [LaTeX]TeX, Bash, Python
\begin{lstlisting}[numbers=left, frame=l, framerule=0.1pt,%
% -----------
	caption={ }, % Caption
	label={code: }  % Label
]% ---------=

  cd projekt
  mkdir -p work neu alt
  # löschen
  rm -rf work alt neu
\end{lstlisting}

\subsubsection{bestehendes Repository
clonen}\label{bestehendes-repository-clonen}

% Quellcode
\lstset{language=Bash} % C, [LaTeX]TeX, Bash, Python
\begin{lstlisting}[numbers=left, frame=l, framerule=0.1pt,%
% -----------
	caption={ }, % Caption
	label={code: }  % Label
]% ---------=

  cd # ? Repository
  git clone . ../work
\end{lstlisting}

\subsubsection{Arbeitsverzeichnis
bearbeiten}\label{arbeitsverzeichnis-bearbeiten}

\textbf{bearbeiten 1}

% Quellcode
\lstset{language=Bash} % C, [LaTeX]TeX, Bash, Python
\begin{lstlisting}[numbers=left, frame=l, framerule=0.1pt,%
% -----------
	caption={ }, % Caption
	label={code: }  % Label
]% ---------=

  cd work
  vi test.md
    # file
    Basis

  # git versionieren
  git add .
  git commit -a
  git status
\end{lstlisting}

\textbf{bearbeiten 2}

% Quellcode
\lstset{language=Bash} % C, [LaTeX]TeX, Bash, Python
\begin{lstlisting}[numbers=left, frame=l, framerule=0.1pt,%
% -----------
	caption={ }, % Caption
	label={code: }  % Label
]% ---------=

  vi test.md
    # file
    Basis
    2) Version

  # git versionieren
  git commit -a
  git status
\end{lstlisting}

\textbf{bearbeiten 3}

% Quellcode
\lstset{language=Bash} % C, [LaTeX]TeX, Bash, Python
\begin{lstlisting}[numbers=left, frame=l, framerule=0.1pt,%
% -----------
	caption={ }, % Caption
	label={code: }  % Label
]% ---------=

  vi test.md
    # file
    Basis
    2) Version
    3) Version

  # git versionieren
  git commit -a
  git status
  git log --graph --oneline
\end{lstlisting}

\subsubsection{Wiederherstellen: Repository in ein temp. Verzeichnis
klonen}\label{wiederherstellen-repository-in-ein-temp.-verzeichnis-klonen}

% Quellcode
\lstset{language=Bash} % C, [LaTeX]TeX, Bash, Python
\begin{lstlisting}[numbers=left, frame=l, framerule=0.1pt,%
% -----------
	caption={ }, % Caption
	label={code: }  % Label
]% ---------=

  cd work
  git clone . ../neu
  git clone . ../alt
\end{lstlisting}

\subsubsection{Wechsel auf den gewünschten
Git-Branch}\label{wechsel-auf-den-gewuenschten-git-branch}

% Quellcode
\lstset{language=Bash} % C, [LaTeX]TeX, Bash, Python
\begin{lstlisting}[numbers=left, frame=l, framerule=0.1pt,%
% -----------
	caption={ }, % Caption
	label={code: }  % Label
]% ---------=

  cd ../neu/
  git stash
  git log --graph --oneline
  * 48eba8f (HEAD -> master, origin/master, origin/HEAD) version3
  * 69383f1 version2
  * 1ef0339 test.md basis
  * 47ac1f2 Projekt init
  # version2
  git reset --hard 69383f1
\end{lstlisting}

\subsubsection{verschiebe .git in den Workspace der alten
Versionsverwaltung}\label{verschiebe-.git-in-den-workspace-der-alten-versionsverwaltung}

% Quellcode
\lstset{language=Bash} % C, [LaTeX]TeX, Bash, Python
\begin{lstlisting}[numbers=left, frame=l, framerule=0.1pt,%
% -----------
	caption={ }, % Caption
	label={code: }  % Label
]% ---------=

  git archive master | tar -x -C ../alt/
\end{lstlisting}

\subsubsection{Ergebnis prüfen}\label{ergebnis-pruefen}

% Quellcode
\lstset{language=Bash} % C, [LaTeX]TeX, Bash, Python
\begin{lstlisting}[numbers=left, frame=l, framerule=0.1pt,%
% -----------
	caption={ }, % Caption
	label={code: }  % Label
]% ---------=

  cd projekt
  kdiff3 alt/ neu/
\end{lstlisting}

\subsection{Repository clonen und von Github
downloaden}\label{repository-clonen-und-von-github-downloaden}

\subsubsection{Ordner für Experimente erstellen -
löschen}\label{ordner-fuer-experimente-erstellen-loeschen-1}

% Quellcode
\lstset{language=Bash} % C, [LaTeX]TeX, Bash, Python
\begin{lstlisting}[numbers=left, frame=l, framerule=0.1pt,%
% -----------
	caption={ }, % Caption
	label={code: }  % Label
]% ---------=

  cd projekt
  mkdir -p lokale-vers github-vers lokale-backup-vers
  # löschen
  rm -rf lokale-vers github-vers lokale-backup-vers
\end{lstlisting}

\subsubsection{lokales Repository}\label{lokales-repository}

HEAD -> master

% Quellcode
\lstset{language=Bash} % C, [LaTeX]TeX, Bash, Python
\begin{lstlisting}[numbers=left, frame=l, framerule=0.1pt,%
% -----------
	caption={ }, % Caption
	label={code: }  % Label
]% ---------=

  cd work
  # repository clonen
  git clone . ../lokale-vers

  # backup
  cd ../lokale-vers
  #tar cvzf ../lokale-vers.tar.gz .
  verz="lokale-vers"
  ID=$(git rev-parse --short HEAD) # git commit (hashwert)
  timestamp=$(date +"%Y-%h-%d_%H:%M") # Datum
  tar cvzf ../"$verz"_"$ID"_"$timestamp".tar.gz .
  cd ..
\end{lstlisting}

\subsubsection{Github Repository}\label{github-repository}

origin/master

% Quellcode
\lstset{language=Bash} % C, [LaTeX]TeX, Bash, Python
\begin{lstlisting}[numbers=left, frame=l, framerule=0.1pt,%
% -----------
	caption={ }, % Caption
	label={code: }  % Label
]% ---------=

  cd github-vers
  # repository clonen
  git clone git@github.com:ju-bw/notizenDummyUbuntu-v02.git .

  # backup
  #tar cvzf ../github-vers.tar.gz .
  verz="github-vers"
  ID=$(git rev-parse --short HEAD) # git commit (hashwert)
  timestamp=$(date +"%Y-%h-%d_%H:%M") # Datum
  tar cvzf ../"$verz"_"$ID"_"$timestamp".tar.gz .
  cd ..
\end{lstlisting}

\subsubsection{lokales backup
Repository}\label{lokales-backup-repository}

backup/master

% Quellcode
\lstset{language=Bash} % C, [LaTeX]TeX, Bash, Python
\begin{lstlisting}[numbers=left, frame=l, framerule=0.1pt,%
% -----------
	caption={ }, % Caption
	label={code: }  % Label
]% ---------=

  cd lokale-backup-vers
  # repository clonen
  git clone /media/jan/virtuell/git-server-repo/notizenDummyUbuntu-v02-backup.git .

  # backup
  #tar cvzf ../lokale-backup-vers.tar.gz .
  verz="lokale-backup-vers"
  ID=$(git rev-parse --short HEAD) # git commit (hashwert)
  timestamp=$(date +"%Y-%h-%d_%H:%M") # Datum
  tar cvzf ../"$verz"_"$ID"_"$timestamp".tar.gz .
  cd ..
\end{lstlisting}

\subsubsection{Ergebnis prüfen}\label{ergebnis-pruefen-1}

% Quellcode
\lstset{language=Bash} % C, [LaTeX]TeX, Bash, Python
\begin{lstlisting}[numbers=left, frame=l, framerule=0.1pt,%
% -----------
	caption={ }, % Caption
	label={code: }  % Label
]% ---------=

  cd projekt
  # verzeichnisse vergleichen
  kdiff3 lokale-vers/ github-vers/ lokale-backup-vers/
  # files vergleichen
  kdiff3 lokale-vers/Readme.md github-vers/Readme.md
\end{lstlisting}

\subsubsection{build - Versionen
erstellen}\label{build-versionen-erstellen}

% Quellcode
\lstset{language=Bash} % C, [LaTeX]TeX, Bash, Python
\begin{lstlisting}[numbers=left, frame=l, framerule=0.1pt,%
% -----------
	caption={ }, % Caption
	label={code: }  % Label
]% ---------=

  cd projekt
  ls -lh *gz
    # Inhalt
    9,3M Apr  7 19:25 github-vers_47ac1f2_2019-Apr-07_19:25.tar.gz
    9,3M Apr  7 18:21 github-vers.tar.gz
    9,4M Apr  7 19:25 lokale-backup-vers_47ac1f2_2019-Apr-07_19:25.tar.gz
    9,4M Apr  7 18:21 lokale-backup-vers.tar.gz
    9,4M Apr  7 19:24 lokale-vers_48eba8f_2019-Apr-07_19:24.tar.gz
    9,4M Apr  7 18:20 lokale-vers.tar.gz
\end{lstlisting}

\subsubsection{build - Versionen
erstellen}\label{build-versionen-erstellen-1}

% Quellcode
\lstset{language=Bash} % C, [LaTeX]TeX, Bash, Python
\begin{lstlisting}[numbers=left, frame=l, framerule=0.1pt,%
% -----------
	caption={ }, % Caption
	label={code: }  % Label
]% ---------=

  file="MD5-Hash.txt"
  printf "# ------------\n"   >  $file
  printf "# build - Versionen   \n"   >> $file
  printf "# lokale-vers:        \n"   >> $file
  printf "# github-vers:        \n"   >> $file
  printf "# lokale-backup-vers: \n"   >> $file
  printf "# Datum:                  \n"   >> $file
  printf "# Git - Hashwert:     \n"   >> $file
  printf "# MD5-Hash:           \n"   >> $file
  printf "# ------------\n\n" >> $file

  # hashwert erstellen
  md5sum github-vers_47ac1f2_2019-Apr-07_19:25.tar.gz        >> $file
  md5sum lokale-backup-vers_47ac1f2_2019-Apr-07_19:25.tar.gz >> $file
  md5sum lokale-vers_48eba8f_2019-Apr-07_19:24.tar.gz        >> $file

  # build - Versionen
  vi MD5-Hash.txt
\end{lstlisting}
%-----------
